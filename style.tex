\usepackage{graphicx}
\graphicspath{{figures/}}
\usepackage{microtype}
\usepackage{csquotes}
\usepackage{standalone}
\usepackage[mathrm=sym]{unicode-math}
\usepackage[binary-units]{siunitx}
\usepackage{datetime2}
\usepackage{booktabs}
\usepackage{libertinus}
\usepackage{multirow}
\usepackage{amsmath}
\usepackage{setspace}
\usepackage{footnote}
\usepackage[inline]{enumitem}
\usepackage{caption}
\usepackage{stackengine}
\usepackage{tabularx}
\usepackage{dsfont}
\usepackage{tcolorbox}
\usepackage{bm}
\usepackage{tikz}
\usepackage{subcaption}
\usepackage{polyglossia}
\usepackage{fontspec}
\setmainlanguage{english}
\usepackage[flushleft]{threeparttable}
\renewcommand{\tabularxcolumn}[1]{m{#1}}
\emergencystretch=1cm
\newcommand{\baselinestretchsize}{1.1}
\renewcommand{\baselinestretch}{\baselinestretchsize}

\usepackage[ruled,linesnumbered]{algorithm2e}
\renewcommand{\floatpagefraction}{0.80}
\renewcommand{\textfraction}{0.1}
\renewcommand{\topfraction}{0.9}
\renewcommand{\bottomfraction}{0.9}

\setcounter{totalnumber}{5}
\setcounter{topnumber}{5}
\setcounter{bottomnumber}{5}

\usepackage{geometry}
\ifthenelse{\boolean{bPrintVersion}}
{
	\geometry{right=2.6cm, textwidth=15cm, top=4.1cm, textheight=21.5cm}
}{
	\geometry{right=3cm, textwidth=15cm, top=4.1cm, textheight=21.5cm}
}

\usepackage{fancyhdr}
\pagestyle{fancy}
\fancyfoot{}

\renewcommand{\chaptermark}[1]{\markboth{\chaptername\ \thechapter.\ #1}{}}
\renewcommand{\sectionmark}[1]{\markright{\thesection.\ #1}}
\fancyhead[LE,RO]{\thepage}
\fancyhead[RE]{\leftmark}
\fancyhead[LO]{\rightmark}
\renewcommand{\headrulewidth}{0.3pt}
\setlength{\headheight}{13.6pt}

%Options: Sonny, Lenny, Glenn, Conny, Rejne, Bjarne, Bjornstrup
\usepackage[Bjornstrup]{fncychap}
% for bjornstrup, to remove overfull hbox warnings
\renewcommand{\DOCH}{%
	\settowidth{\py}{\CNoV\thechapter}
	\addtolength{\py}{-10pt}      % Amount of space by which the number is shifted right
	\fboxsep=0pt%
	\colorbox[gray]{.85}{\rule{0pt}{40pt}\parbox[b]{\textwidth}{\hfill}}%
	\kern-\py\raise20pt%
	\rlap{\color[gray]{.5}\CNoV\thechapter}\\%
}
\setlength{\marginparwidth}{2cm}
\usepackage{todonotes}
\newcommand{\insertfigure}[2][]{
	\begin{figure}
		\missingfigure[#1]{#2}
		\caption{#2}
	\end{figure}
}
\newcommand{\insertref}[1]{\todo[fancyline,color=green!40]{Add reference: #1}}
\newcommand{\explainindetail}[1]{\todo[fancyline,color=red!40]{#1}}
\newcommand{\inlinetodo}[1]{\todo[inline]{#1}}
\newcommand{\generictodo}[1]{\todo[fancyline,color=green!40]{#1}}

\ifthenelse{\boolean{bDraft}}
{
	\voffset = 0cm
	\newcommand{\drafttext}{\emph{Draft Copy: \today\\}}
	\fancyfoot[C]{\drafttext}
	\fancypagestyle{plain}{%
		\fancyhf{}%
		\fancyfoot[C]{\drafttext}%
		\renewcommand{\headrulewidth}{0pt}%
		\renewcommand{\footrulewidth}{0pt}%
	}
}{}

\ifthenelse{\boolean{bPrintVersion}}
{
	\usepackage[
		unicode=true,
		colorlinks  = true,
		citecolor   = black,
		linkcolor   = black,
		urlcolor    = black,
		breaklinks  = true,
		pdfauthor   = {Martin\ Zach},
		pdftitle    = {Generative\ Regularizers\ in\ Computed\ Tomography\ (print-version)},
		pdfsubject  = {Development\ and\ Application\ of\ a\ Generative\ Regularizer\ to\ be\ used\ in\ Computed\ Tomography.},
		pdfkeywords = {Computer\ Vision;\ Computed\ Tomography;\ Maximum-Likelihood;\ Generative\ Models;\ Variational\ Methods},
		pdfcreator  = {LaTeX\ on\ Arch\ Linux\ and\ neovim\ with\ VimTeX},
		pdfproducer = {xelatex}
	]{hyperref}
}{
	\usepackage[
		unicode=true,
		colorlinks  = true,
		citecolor   = blue,
		linkcolor   = blue,
		urlcolor    = black,
		breaklinks  = true,
		pdfauthor   = {Martin\ Zach},
		pdftitle    = {Generative\ Regularizers\ in\ Computed\ Tomography\ (web-version)},
		pdfsubject  = {Development\ and\ Application\ of\ a\ Generative\ Regularizer\ to\ be\ used\ in\ Computed\ Tomography.},
		pdfkeywords = {Computer\ Vision;\ Computed\ Tomography;\ Maximum-Likelihood;\ Generative\ Models;\ Variational\ Methods},
		pdfcreator  = {LaTeX\ on\ Arch\ Linux\ and\ neovim\ with\ VimTeX},
		pdfproducer = {xelatex}
	]{hyperref}
}

\usepackage[
	hyperref=true,
	backref=true,
	backrefstyle=three,
	url=false,
	eprint=false,
	giveninits=true,
]{biblatex}
\addbibresource{bibliography.bib}
\DefineBibliographyStrings{english}{%
	backrefpage = {cited on page},%
	backrefpages = {cited on pages},%
}
\setcounter{secnumdepth}{3}
\setcounter{tocdepth}{3}

% Clear Header Style on the Last Empty Odd pages
\makeatletter
\def\cleardoublepage{\clearpage\if@twoside \ifodd\c@page\else%
\hbox{}%
\thispagestyle{empty}%
\newpage%
\if@twocolumn\hbox{}\newpage\fi\fi\fi}
\makeatother

\usepackage[nohints]{minitoc}
\setcounter{minitocdepth}{1}
\mtcsetfeature{minitoc}{after}{\vspace*{0.5cm}}
\usepackage{placeins}

\usepackage{epigraph}
\renewcommand{\textflush}{flushepinormal}
\setlength{\epigraphwidth}{8cm}
\newcommand{\myepigraph}[3]{\epigraph{#1}{#2, \textit{#3}}}
\newcommand{\transexpl}[3]{#1 (#2, #3)}
\usepackage[capitalize]{cleveref}

\usetikzlibrary{angles,quotes,positioning}
\usepackage{pgfplots}
\usepackage{siunitx}
\definecolor{light}{HTML}{bdbdbd}
\definecolor{dark}{HTML}{191919}

\pgfplotsset{compat=1.17}
\usepgfplotslibrary{
  groupplots
}
\usetikzlibrary{matrix}
\usetikzlibrary{shapes.arrows}
\tikzset{
	source/.pic = {
		\draw (0.5, 0.7) -- ++(0, -0.2) --
		      ++(-0.5, 0) -- ++(0, 0.5) --
		      ++(-0.7, 0) -- ++(0, -2) --
		      ++(0.7, 0) -- ++(0, 0.5) --
		      ++(0.5, 0) -- ++(0, -0.2);
	},
	detector/.pic = {
		\draw [fill=gray] (-1, 1) rectangle (1, -1);
		\draw (-0.8, 1) -- ++(0, -2);
	},
	slab/.pic = {
		\draw [fill=gray!50] (-0.2, 2) rectangle (0.2, -2);
	},
	human/.pic = {
		\draw [fill=gray] (0, 0) 
		.. controls ++(180:-1) and ++(-90: 1) .. ( 2, 1)
		.. controls ++(-90:-1) and ++(180:-1) .. ( 0, 2)
		.. controls ++(180: 1) and ++(-90:-1) .. (-2, 1)
		.. controls ++(-90: 1) and ++(180: 1) .. ( 0, 0);
		\draw [fill=white] (0, 1.5) ellipse (0.8 and 1);
		\draw (-.1, 2.49) to [in=180, out=30] ++(0.1, 0.2) to [in=150, out=0] ++(0.1, -0.2);
		\path[thick, fill=white] (-.1, 2.48) to [in=180, out=30] ++(0.1, 0.2) to [in=150, out=0] ++(0.1, -0.2) --cycle;
	}
}

\usepackage{tikz}
\usepackage{xcolor}
\usetikzlibrary{arrows,decorations.pathmorphing,shapes,calc}

\definecolor{photoncolor}{rgb}{0.65, 0.16, 0.16}
\definecolor{electroncolor}{rgb}{0.65, 0.65, 0.65}
%\proton{xposition,yposition}
\newcommand{\proton}[1]{%
	\shade [ball color=black!60] (#1) circle (.25);
	\draw (#1) node{$+$};
}

%\neutron{xposition,yposition}
\newcommand{\neutron}[1]{%
	\shade[ball color=black!20] (#1) circle (.25);
}

%\electron{xwidth,ywidth,rotation angle}
\newcommand{\electron}[3]{%
	\draw [rotate = #3] (0, 0) ellipse (#1 and #2) [color=gray];
	\shade [ball color=electroncolor] (0, #2) [rotate=#3] circle (.1);
}

\newcommand{\nucleus}{%
	\neutron{0.1,0.3}
	\proton{0,0}
	\neutron{0.3,0.2}
	\proton{-0.2,0.1}
	\neutron{-0.1,0.3}
	\proton{0.2,-0.15}
	\neutron{-0.05,-0.12}
	\proton{0.17,0.21}
}

%\photoelectron{xwidth,ywidth,rotation angle}
\newcommand{\photoelectron}[3]{%
	\draw [rotate=#3] (0, 0) ellipse (#1 and #2) [dashed, color=gray];%
	\draw [
		-latex, decorate, decoration={snake, amplitude=0.3cm, segment length=20pt, post length=1mm},%
		color=photoncolor
	] ($(#3:#1) + (-6,0)$) -- ++(6-0.1, 0);%
	\draw [-stealth] (#3:#1) -- ++(5, 0.625);
	\shade [ball color=electroncolor] (#3:#1) -- ++(4, 0.5) node [text width=1.8cm, above=0.2cm] {\large \( e^- \) from an inner shell} circle(0.1);
	\draw (#1,0) [rotate=#3, color=electroncolor, fill=white] circle (.1);
	\draw [
		-latex, decorate, decoration={snake, amplitude=0.3cm, segment length=50pt, post length=1mm},%
		color=photoncolor
	] (1, -1) -- ++(6, -2);%
	\node [rotate=-24, text width=3cm] at (6, -1.7) {\large Characteristic Radiation};
}

%\comptonelectron{xwidth,ywidth,rotation angle}
\newcommand{\comptonelectron}[3]{%
	\draw [rotate=#3] (0,0) ellipse (#1 and #2) [dashed, color=gray];
	\draw [
		-latex, decorate, decoration={snake, amplitude=0.3cm, segment length=0.5cm, post length=1mm},
		color=photoncolor
	] ($(#3:#1) + (-6, 0)$) -- ++(6 - 0.1, 0);
	\draw [gray, dashdotted] ($(#3:#1) + (-6, 0)$) -- ++(10, 0);
	\draw [gray, dashdotted] (#3:#1) ++(0:3) arc (0:-26:3) node [black, midway, left] {\large \( \theta \)};
	\draw [-stealth] (#3:#1) -- ++(5, 2.5);
	\shade [ball color=electroncolor] (#3:#1)  --  ++(4,2.0) node[text width=2cm, below right=0.05cm]{\large \( e^- \) from an outer shell} circle(0.1);
	\draw (#1,0) [rotate=#3, color=electroncolor, fill=white] circle (.1) ;
	\draw [
		decorate, decoration={snake, amplitude=0.3cm, post length=1mm, segment length=0.7cm},
		color=photoncolor,-stealth
	] ($(#3:#1) + (0.1, 0)$) -- ++(6, -3);
	\node [rotate=-26] at (5.5, 1.5) {\large Compton photon};%
	\draw [gray, dashdotted] ($(#3:#1) + (0.1, 0)$) -- ++(3.5, -1.75);
}

%\paircreation{impact parameter}
\newcommand{\paircreation}[1]{%
	\draw[
		-latex, decorate,
		decoration={snake, amplitude=0.3cm, segment length=0.3cm, post length=2mm},
		color=photoncolor
	] (-6, #1) -- ++(6, 0);
	\draw [-stealth] (0, #1) -- ++ (5, 2.5);
	\draw [dashed, fill=white] (4, #1 + 2.0) node[below right=0.1cm] {\large \( e^+ \)} circle (0.1);
	\draw [-stealth] (0, #1) -- ++ (4, -2.0);
	\shade [ball color=gray] (0, #1) -- ++(3.2, -1.6) node[above right=0.1cm] {\large \( e^- \)} circle(0.1);
}

\tikzset{
	source/.pic = {
		\draw (0.5, 0.7) -- ++(0, -0.2) --
		      ++(-0.5, 0) -- ++(0, 0.5) --
		      ++(-0.7, 0) -- ++(0, -2) --
		      ++(0.7, 0) -- ++(0, 0.5) --
		      ++(0.5, 0) -- ++(0, -0.2);
	},
	detector/.pic = {
		\draw [fill=gray] (-1, 1) rectangle (1, -1);
		\draw (-0.8, 1) -- ++(0, -2);
	},
	slab/.pic = {
		\draw [fill=gray!50] (-0.2, 2) rectangle (0.2, -2);
	},
	human/.pic = {
		\draw [fill=gray] (0, 0) 
		.. controls ++(180:-1) and ++(-90: 1) .. ( 2, 1)
		.. controls ++(-90:-1) and ++(180:-1) .. ( 0, 2)
		.. controls ++(180: 1) and ++(-90:-1) .. (-2, 1)
		.. controls ++(-90: 1) and ++(180: 1) .. ( 0, 0);
		\draw [fill=white] (0, 1.5) ellipse (0.8 and 1);
		\draw (-.1, 2.49) to [in=180, out=30] ++(0.1, 0.2) to [in=150, out=0] ++(0.1, -0.2);
		\path[thick, fill=white] (-.1, 2.48) to [in=180, out=30] ++(0.1, 0.2) to [in=150, out=0] ++(0.1, -0.2) --cycle;
	}
}


\usetikzlibrary{calc, positioning, shapes, backgrounds, fit, arrows, spy}
\usepackage{pgf-spectra}
\usepackage{siunitx}
\usepackage{tikz-3dplot}
\definecolor{tomplane}{rgb}{1,0,0}
\definecolor{projplane}{rgb}{0,0,1}
\definecolor{coordsys}{rgb}{0.5,0.5,0}
\usetikzlibrary{shadows}
\usetikzlibrary{calc}

\makeatletter

% Initialize H matrix for perspective view
\pgfmathsetmacro\H@tpp@aa{1}\pgfmathsetmacro\H@tpp@ab{0}\pgfmathsetmacro\H@tpp@ac{0}%\pgfmathsetmacro\H@tpp@ad{0}
\pgfmathsetmacro\H@tpp@ba{0}\pgfmathsetmacro\H@tpp@bb{1}\pgfmathsetmacro\H@tpp@bc{0}%\pgfmathsetmacro\H@tpp@bd{0}
\pgfmathsetmacro\H@tpp@ca{0}\pgfmathsetmacro\H@tpp@cb{0}\pgfmathsetmacro\H@tpp@cc{1}%\pgfmathsetmacro\H@tpp@cd{0}
\pgfmathsetmacro\H@tpp@da{0}\pgfmathsetmacro\H@tpp@db{0}\pgfmathsetmacro\H@tpp@dc{0}%\pgfmathsetmacro\H@tpp@dd{1}

%Initialize H matrix for main rotation
\pgfmathsetmacro\H@rot@aa{1}\pgfmathsetmacro\H@rot@ab{0}\pgfmathsetmacro\H@rot@ac{0}%\pgfmathsetmacro\H@rot@ad{0}
\pgfmathsetmacro\H@rot@ba{0}\pgfmathsetmacro\H@rot@bb{1}\pgfmathsetmacro\H@rot@bc{0}%\pgfmathsetmacro\H@rot@bd{0}
\pgfmathsetmacro\H@rot@ca{0}\pgfmathsetmacro\H@rot@cb{0}\pgfmathsetmacro\H@rot@cc{1}%\pgfmathsetmacro\H@rot@cd{0}
%\pgfmathsetmacro\H@rot@da{0}\pgfmathsetmacro\H@rot@db{0}\pgfmathsetmacro\H@rot@dc{0}\pgfmathsetmacro\H@rot@dd{1}

\pgfkeys{
    /three point perspective/.cd,
        p/.code args={(#1,#2,#3)}{
            \pgfmathparse{int(round(#1))}
            \ifnum\pgfmathresult=0\else
                \pgfmathsetmacro\H@tpp@ba{#2/#1}
                \pgfmathsetmacro\H@tpp@ca{#3/#1}
                \pgfmathsetmacro\H@tpp@da{ 1/#1}
                \coordinate (vp-p) at (#1,#2,#3);
            \fi
        },
        q/.code args={(#1,#2,#3)}{
            \pgfmathparse{int(round(#2))}
            \ifnum\pgfmathresult=0\else
                \pgfmathsetmacro\H@tpp@ab{#1/#2}
                \pgfmathsetmacro\H@tpp@cb{#3/#2}
                \pgfmathsetmacro\H@tpp@db{ 1/#2}
                \coordinate (vp-q) at (#1,#2,#3);
            \fi
        },
        r/.code args={(#1,#2,#3)}{
            \pgfmathparse{int(round(#3))}
            \ifnum\pgfmathresult=0\else
                \pgfmathsetmacro\H@tpp@ac{#1/#3}
                \pgfmathsetmacro\H@tpp@bc{#2/#3}
                \pgfmathsetmacro\H@tpp@dc{ 1/#3}
                \coordinate (vp-r) at (#1,#2,#3);
            \fi
        },
        coordinate/.code args={#1,#2,#3}{
            \def\tpp@x{#1}
            \def\tpp@y{#2}
            \def\tpp@z{#3}
        },
}

\tikzset{
    view/.code 2 args={
        \pgfmathsetmacro\rot@main@theta{#1}
        \pgfmathsetmacro\rot@main@phi{#2}
        % Row 1
        \pgfmathsetmacro\H@rot@aa{cos(\rot@main@phi)}
        \pgfmathsetmacro\H@rot@ab{sin(\rot@main@phi)}
        \pgfmathsetmacro\H@rot@ac{0}
        % Row 2
        \pgfmathsetmacro\H@rot@ba{-cos(\rot@main@theta)*sin(\rot@main@phi)}
        \pgfmathsetmacro\H@rot@bb{cos(\rot@main@phi)*cos(\rot@main@theta)}
        \pgfmathsetmacro\H@rot@bc{sin(\rot@main@theta)}
        % Row 3
        \pgfmathsetmacro\H@m@ca{sin(\rot@main@phi)*sin(\rot@main@theta)}
        \pgfmathsetmacro\H@m@cb{-cos(\rot@main@phi)*sin(\rot@main@theta)}
        \pgfmathsetmacro\H@m@cc{cos(\rot@main@theta)}
        % Set vector values
        \pgfmathsetmacro\vec@x@x{\H@rot@aa}
        \pgfmathsetmacro\vec@y@x{\H@rot@ab}
        \pgfmathsetmacro\vec@z@x{\H@rot@ac}
        \pgfmathsetmacro\vec@x@y{\H@rot@ba}
        \pgfmathsetmacro\vec@y@y{\H@rot@bb}
        \pgfmathsetmacro\vec@z@y{\H@rot@bc}
        % Set pgf vectors
        \pgfsetxvec{\pgfpoint{\vec@x@x cm}{\vec@x@y cm}}
        \pgfsetyvec{\pgfpoint{\vec@y@x cm}{\vec@y@y cm}}
        \pgfsetzvec{\pgfpoint{\vec@z@x cm}{\vec@z@y cm}}
    },
}

\tikzset{
    perspective/.code={\pgfkeys{/three point perspective/.cd,#1}},
    perspective/.default={p={(15,0,0)},q={(0,15,0)},r={(0,0,50)}},
}

\tikzdeclarecoordinatesystem{three point perspective}{
    \pgfkeys{/three point perspective/.cd,coordinate={#1}}
    \pgfmathsetmacro\temp@p@w{\H@tpp@da*\tpp@x + \H@tpp@db*\tpp@y + \H@tpp@dc*\tpp@z + 1}
    \pgfmathsetmacro\temp@p@x{(\H@tpp@aa*\tpp@x + \H@tpp@ab*\tpp@y + \H@tpp@ac*\tpp@z)/\temp@p@w}
    \pgfmathsetmacro\temp@p@y{(\H@tpp@ba*\tpp@x + \H@tpp@bb*\tpp@y + \H@tpp@bc*\tpp@z)/\temp@p@w}
    \pgfmathsetmacro\temp@p@z{(\H@tpp@ca*\tpp@x + \H@tpp@cb*\tpp@y + \H@tpp@cc*\tpp@z)/\temp@p@w}
    \pgfpointxyz{\temp@p@x}{\temp@p@y}{\temp@p@z}
}
\tikzaliascoordinatesystem{tpp}{three point perspective}

\makeatother
\colorlet{myblue}{black!40!blue}
\colorlet{myred}{black!40!red}

\definecolor{coilcolor}{rgb}{0.8,0.5,0.5}
\definecolor{photoncolor}{rgb}{0.65, 0.16, 0.16}
\definecolor{secondary}{RGB}{181, 137, 0}
\usetikzlibrary{decorations.pathmorphing,calc}


\newcommand{\AxisRotator}[1][rotate=0]{%
    \tikz [x=0.25cm,y=0.60cm,line width=.1ex,-stealth,#1] \draw (0,0) arc (-150:150:1 and 1);%
}
\usetikzlibrary{decorations.pathreplacing,decorations.pathmorphing,calc}
\def\centerarc[#1](#2)(#3:#4:#5)% Syntax: [draw options] (center) (initial angle:final angle:radius)
    { \draw[#1] ($(#2)+({#5*cos(#3)},{#5*sin(#3)})$) arc (#3:#4:#5); }
\pgfmathdeclarefunction{gauss}{3}{%
  \pgfmathparse{1/(#3*sqrt(2*pi))*exp(-((#1-#2)^2)/(2*#3^2))}%
}
\usepgfplotslibrary{fillbetween}
\definecolor{normalanno}{HTML}{446699}
\definecolor{threebythree}{HTML}{543639}
\newcommand{\networkblock}[3]{%
	\pgfmathsetmacro{\cubex}{#1}
	\pgfmathsetmacro{\cubey}{#2}
	\pgfmathsetmacro{\cubez}{#3}
	\draw[line join=bevel, ultra thick,draw=photoncolor,fill=light] (0,0,0) -- ++(-\cubex,0,0) -- ++(0,-\cubey,0) -- ++(\cubex,0,0) -- cycle;
	\draw[line join=bevel, ultra thick,draw=photoncolor,fill=light] (0,0,0) -- ++(0,0,-\cubez) -- ++(0,-\cubey,0) -- ++(0,0,\cubez) -- cycle;
	\draw[line join=bevel, ultra thick,draw=photoncolor,fill=light] (0,0,0) -- ++(-\cubex,0,0) -- ++(0,0,-\cubez) -- ++(\cubex,0,0) -- cycle;
}
