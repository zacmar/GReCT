% !TEX root = PhD_main.tex
\chapter*{Abstract}
%\renewcommand{\baselinestretch}{1.3}\normalsize

In today's medical landscape, imaging systems are of exceptional importance.
Magnetic Resonance Imaging and \gls{ct} are heavily used and have improved diagnostic capabilities in many fields.
However, contrast in \gls{ct} relies on depositing ionizing radiation in the body, the dose of which should be as low as possible.
Image quality per dose has improved drastically in the past decades, with advances in instrumentation and reconstruction algorithms.

In this work, we continue this trend by introducing a novel regularization scheme, where a parametrized regularizer is learned on data using maximum likelihood.
Our energy-based formulation allows for much improved interpretability when compared to traditional feed-forward approaches.
We can draw samples from our prior as well as the posterior of any given reconstruction problem, such that domain experts can judge the regularizer.
We apply the regularizer to typical reconstruction tasks such as limited-angle and few-view \gls{ct} reconstruction.
Our model outperforms traditional reconstruction algorithms by a large margin.
\paragraph{Keywords.} Deep Learning, Variational Methods, Data-driven Regularizers, Maximum Likelihood, Computed Tomography, Inverse Problems
