\documentclass[../PhD_main.tex]{subfiles}
\begin{document}


\newcommand{\textcommand}[2]{\textbackslash{#1}\textbraceleft #2\textbraceright}


\chapter{HowTo}
\label{chap:howto}


\begin{quote}
	You can start your chapter with a reference to a publication, \eg:
	
	This chapter is based on my awesome publication presented at CVPR.
	
	\fullcite{MyAwesomePublication2019}
\end{quote}

\minitoc

\section{Style}
\label{sec:howto:intro_contributions}
The main usepackages are defined in the file \texttt{style.tex}. Here, also settings to produce the pdf file are set. You have to change title and author there to display the correct information in the pdf.

\section{Todos and comments}
You can add todos using the \texttt{todo} command \todo{add some todo} or add some comments using \texttt{mycomment}. Both commands produce an entry in the todo list.

\section{Referencing}
\label{sec:howto:referencing}
This thesis now uses the \texttt{cleveref} package. There is no need to write \texttt{Eq.\textasciitilde\textcommand{ref}{eq:howto:myeq}} or \texttt{Fig.\textasciitilde \textcommand{ref}{fig:howto:myfig}} anymore.
Just use the \texttt{\textbackslash{cref}} command and it is displayed automatically!
Just make sure to correctly and uniquely label your figures, tables, equations, sections, chapters,...
More information on the cleveref package can be found in the \href{http://tug.ctan.org/tex-archive/macros/latex/contrib/cleveref/cleveref.pdf}{package documentation}.
\\

\noindent Examples using \texttt{\textbackslash{cref}}:
\begin{itemize}
	\item \texttt{\textcommand{cref}{fig:howto:myfig}} displays \cref{fig:howto:myfig}
	\item To reference at the start of a sentence use \texttt{\textcommand{Cref}{fig:howto:myfig}} that displays \Cref{fig:howto:myfig}
	\item \texttt{\textcommand{cref}{eq:howto:myeq}} displays \cref{eq:howto:myeq}
	\item \texttt{\textcommand{cref}{sec:howto:referencing}} displays \cref{sec:howto:referencing}
\end{itemize}

\begin{equation}\label{eq:howto:myeq}
a^2 + b^2 = c^2
\end{equation}

\begin{figure}
\centering
\includegraphics[width=2cm]{empty_image}
\caption{myfig}\label{fig:howto:myfig}
\end{figure}

\section{Glossary}
\label{sec:howto:glossary}
This template uses the \texttt{glossaries} package. New abbreviations can be added to the file \texttt{./def/def\_acronyms\_abbrevs.tex}. The abbreviations are fully defined at first use, afterwards only the abbreviations are shown.

\begin{itemize}
	\item \texttt{\textcommand{gls}{ai}} displays \gls{ai}
	\item Now, \texttt{\textcommand{gls}{ai}} displays \gls{ai}
	\item The plural form can be displayed with \texttt{\textcommand{glspl}{cnn}} displays \glspl{cnn}
\end{itemize}

\section{Bibliography}
You can use different bibliography styles, \eg \texttt{apalike}, or a modified \texttt{plain} defined in \texttt{phdplain}, which includes URLs in the bibliography.
To display single bibentries, you can use the \texttt{bibentry} command, \eg \texttt{\textcommand{bibentry}{MyAwesomePublication2019}} displays\\
\fullcite{MyAwesomePublication2019}
\end{document}
